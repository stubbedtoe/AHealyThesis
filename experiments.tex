\documentclass[]{article}

%I'd like to know a bit more about the experiments - I guess these are
%sections 5.2/3/4.  Would you be able to give me a brief summary of
%these, say a page each: the RQ being addressed, the methodology, and
%what the results would look like (metrics, tables, graphs etc.).

%opening
\title{Thesis Experiments}
\author{Andrew Healy}

\begin{document}

\maketitle

For the following experiments, the provers under consideration are Z3, Alt-Ergo, CVC3, CVC4, Yices, veriT. The sets \emph{training} and \emph{test} are disjoint subsets of the 919 individual proof obligations generated by Why3 from the 117 example WhyML programs included in the standard Why3 distribution. The independent/predictor variables are various metrics statically derived from the proof obligation goals: number of operators, number of variables, number of constants etc. The dependent/repsponse variables are the \emph{time} taken by each prover (measured in seconds) and the \emph{result} returned by Why3. These were dynamically measured with a statistical confidence interval of 90\%. The ratio of the size of the training set to the test set is approximately 3:1.
\vspace{0.4cm}

\textit{Null hypothesis:}

\textbf{Static metrics derived from proof obligation goals do not indicate the performance of any SMT prover.} 

\section{Predicting Prover Result}


Can the correct prover result $r \in \lbrace Valid, Invalid, Unknown, Timeout, Error \rbrace $ be predicted for an arbitrary proof obligation $p \in training$ for each prover?

\subsection{Research Questions addressed}

\begin{itemize}
\item{Can the single most effective solver for a Why3 proof obligation be predicted by learning from static metrics?}
\item{Can a useful ranking of solvers be predicted for a Why3 proof obligation?}
\end{itemize}

\subsection{Methodology}

Whether the procedure below is implemented separately for each prover depends on whether the learning algorithm supports multivariate output (e.g. Decision Trees) or not (e.g Support Vector classifiers). 

\begin{enumerate}
\item{Perform standard scaling of the independent/predictor variables in \emph{training}. Call this matrix $X$}
\item{Separate \textit{result} columns from \textit{time} of the dependent/response variables in \textit{training}. Call this matrix $y$}
\item{Fit the classification model on $X$ and $y$}
\item{For each proof $p$ obligation in \textit{testing}:}
\begin{itemize}
	\item[a.] Scale $p$'s independent/predictor variables with the scaler used in step 1.
	\item[b.] Predict the result $r$ on the model
\end{itemize}
\end{enumerate}
  

\end{document}
